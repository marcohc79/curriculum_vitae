\documentclass[a4paper,10pt]{article}
\usepackage[utf8]{inputenc}
\usepackage[spanish,provide=*]{babel}
\usepackage{geometry}
\usepackage{hyperref}

\geometry{top=2cm, bottom=2cm, left=2.5cm, right=2.5cm}

\begin{document}

\begin{center}
    \textbf{\Huge Marco Huamani} \\
    \vspace{0.2cm}
    Buenos Aires, Argentina · \href{https://www.linkedin.com/in/marco-huamani}{linkedin.com/in/marco-huamani} \\
    15-2255-9494 · \href{mailto:mhcaballero79@gmail.com}{mhcaballero79@gmail.com}
\end{center}

\vspace{0.5cm}

\section*{Resumen}
Soy estudiante de Ingeniería en Informática con experiencia práctica en soporte técnico, administración de sistemas y operaciones de data center. A lo largo de mi formación académica y mi experiencia laboral, he desarrollado un enfoque analítico para resolver problemas complejos, siempre con una mentalidad orientada a resultados. 

Mis conocimientos abarcan lenguajes de programación como C, Python, Java y SQL, aplicados tanto en el desarrollo de software como en la administración de bases de datos. Además, tengo experiencia en la gestión de incidentes y la optimización de procesos operativos en entornos dinámicos. 

Me apasiona trabajar en proyectos que me desafíen y que impliquen innovación tecnológica. Busco contribuir con soluciones efectivas y escalables en un entorno de trabajo colaborativo y en constante evolución.

\vspace{0.5cm}

\section*{Experiencia Profesional}

\subsection*{\Large\textbf{RH-T / NCR}}
\textbf{\normalsize Analista de Monitoreo de Data Center} \hfill Noviembre 2024 – Actualidad
\begin{itemize}
    \item Seguimiento de incidentes reportados por Cencosud, Carrefour.
    \item Estandarización de procesos de monitoreo.
    \item Interacción constante con el equipo de soporte regional en LATAM.
\end{itemize}

\subsection*{\Large\textbf{Tecval S.A.}}
\textbf{\normalsize Operador de Data Center} \hfill Octubre 2021 – Actualidad
\begin{itemize}
    \item Monitoreo de alertas en Zabbix, Grafana y JobScheduler.
    \item Altas, bajas y modificaciones de backups en HP DataProtector.
    \item Gestión de incidentes críticos: creación y coordinación.
    \item Mantenimiento y monitoreo de procesos de carga diaria en operaciones del Merval.
\end{itemize}

\subsection*{\Large\textbf{Tecval S.A.}}
\textbf{\normalsize Microinformático} \hfill Diciembre 2018 – Octubre 2021
\begin{itemize}
    \item Administración de usuarios en Active Directory.
    \item Migración y actualización de PCs y notebooks.
    \item Soporte a usuarios del Grupo BYMA y Alycs.
    \item Instalación y configuración de programas requeridos.
    \item Rackeo y cableado de servidores.
\end{itemize}

\subsection*{\Large\textbf{4Latam}}
\textbf{\normalsize Microinformático} \hfill Junio 2018 – Diciembre 2018
\begin{itemize}
    \item Soporte a usuarios de Samsung S.A.
    \item Gestión de incidentes mediante ticketeras.
    \item Migración y actualización de PCs y notebooks.
    \item Mantenimiento de salas de reunión.
\end{itemize}

\subsection*{\Large\textbf{YelInformático}}
\textbf{\normalsize Microinformático} \hfill Febrero 2018 – Junio 2018
\begin{itemize}
    \item Soporte a usuarios de Swiss Medical Group.
    \item Gestión de incidentes mediante ticketeras.
    \item Migración y actualización de PCs y notebooks.
    \item Guardias en clínicas de CABA.
\end{itemize}

\subsection*{\Large\textbf{Megatech}}
\textbf{\normalsize HelpDesk} \hfill Mayo 2016 – Febrero 2018
\begin{itemize}
    \item Soporte telefónico a usuarios de DrSpeedy.
\end{itemize}

\vspace{0.5cm}

\section*{Educación}

\noindent \textbf{Ingeniería en Informática} \hfill Marzo 2020 – Actualidad \\
Facultad de Ingeniería, UBA \hfill CABA

\vspace{0.5cm}

\noindent \textbf{Técnico en Informática con orientación en redes} \hfill Septiembre 2006 – Diciembre 2008 \\
EET N° 4 \hfill Berazategui \\
Promedio: 8.5

\vspace{0.5cm}

\section*{Habilidades Técnicas}
\begin{itemize}
    \item Lenguajes: C, Python, Java, SQL.
    \item Bases de Datos: Relacionales (MySQL) y No Relacionales.
    \item Herramientas: \href{https://github.com/marcohc79} {GIT}, \LaTeX, Emacs, Neovim.
    \item Sistemas Operativos: Linux, Windows, macOS.
    \item Frameworks: Spring Boot.
    \item Otros: HTML, CSS, IntelliJ, VSCode, LazyVim.
\end{itemize}

\vspace{0.5cm}

\section*{Certificados}
\begin{itemize}
    \item \href{https://todocodeacademy.com/certificate/gav/}{Desarrollo de APIs en Java con SpringBoot — TodoCode (2024)}
    \item \href{https://todocodeacademy.com/certificate/siq/}{Programación Orientada a Objetos con Java — TodoCode (2024)}
    \item \href{https://todocodeacademy.com/certificate/uzh/}{Introducción a las Bases de Datos Relacionales (Con MySQL) — TodoCode (2024)}
    \item \href{https://todocodeacademy.com/certificate/lam/}{Microservicios con Spring Cloud — TodoCode (2024)}
\end{itemize}

\vspace{0.5cm}

\section*{Idiomas}
\begin{itemize}
    \item Inglés: Intermedio.
\end{itemize}

\vspace{0.5cm}

\end{document}
